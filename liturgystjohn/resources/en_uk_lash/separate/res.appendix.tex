\itTopic{en}{uk}{lash}{appendix}
\itKey{ltType}{
resource
}%
\itKey{ltTopic}{
appendix
}%
\itKey{ltDomain}{
en_uk_lash
}%
\itKey{p0001}{
duplicates.d1212
}%
\itKey{p0002}{
duplicates.d0047
}%
\itKey{p0003}{
VESPERS ON ORDINARY DAYS
}%
\itKey{p0004}{
On ordinary weekdays Vespers is celebrated as follows:
}%
\itKey{p0005}{
The Priest reads the seven prayers outside the Sanctuary and remains there for the Litany of Peace. He should remain outside the Sanctuary for the whole service, except for the censing of the church.
}%
\itKey{p0006}{
He enters the Sanctuary at ‘Lord, I have cried’ for the censing of the Church. This is done as usual, but the Priest vests only in the Epitrachelion.
}%
\itKey{p0007}{
Only six stichera are sung at ‘Lord, I have cried’. Normally 3 from the Paraklitiki and 3 from the Menaion. In this case it is the custom in many places to sing 2 from the first set in the Paraklitiki and 1 from the second. If the Saint of the day has a Doxastikon, then 6 Stichera are sung from the Menaion, the 3 appointed being doubled. In this case the Theotokion is the one from the Menaion.
}%
\itKey{p0008}{
There is no Entrance, but immediately after the hymn ‘O Joyful Light’ the Priest announces ‘Prokeimenon of the evening’.
}%
\itKey{p0009}{
The Prokeimenon is followed at once by the prayer ‘Grant, Lord’ and the Litany is displaced until later.
}%
\itKey{p0010}{
The verses for the Aposticha are the following:
}%
\itKey{p0011}{
From Sunday to Thursday
}%
\itKey{p0012}{
Verse 1. To you I lift up my eyes, to you who are enthroned in the heavens. As the eyes of servants look to the hand of their master: or as the eyes of a maid toward the hand of her mistress, so our eyes look to the Lord our God: until he show us his mercy.
}%
\itKey{p0013}{
Verse 2. Have mercy on us, O Lord, have mercy upon us: for we have our fill of derision; our soul has its fill. Mockery for those at ease: and derision for the proud.
}%
\itKey{p0014}{
On Friday.
}%
\itKey{p0015}{
Verse 1: God is wonderful in his Saints.
}%
\itKey{p0016}{
Verse 2: For the Saints in his land the Lord has done wonders.
}%
\itKey{p0017}{
Verse 3: Blessed are those whom you have chosen and taken; they will dwell in your courts.
}%
\itKey{p0018}{
The Litany of Fervent Supplication, omitting the first two petitions, follows the Apolytikion and its Theotokion.
}%
\itKey{p0019}{
The Dismissal is as follows:
}%
\itKey{p0020}{
On Sunday evening
}%
\itKey{p0021}{
May Christ our true God, at the prayers of his most pure and holy Mother; the protection of the honoured Bodiless Powers of heaven; the intercessions of the holy, glorious and all-praised Apostles; of Saint N. [the patron of the church]; [of Saint N., whose memory we celebrate, of the holy and righteous forebears of God, Joachim and Anne;] and of all the Saints, have mercy on us and save us, for he is good and loves mankind.
}%
\itKey{p0022}{
On Monday
}%
\itKey{p0023}{
May Christ our true God, at the prayers of his most pure and holy Mother; the intercessions of the honoured, glorious, prophet, forerunner and Baptist John; of the holy, glorious and all-praised Apostles; of Saint N. [the patron of the church]; [of Saint N., whose memory we celebrate, of the holy and righteous forebears of God, Joachim and Anne;] and of all the Saints, have mercy on us and save us, for he is good and loves mankind.
}%
\itKey{p0024}{
On Tuesday and Thursday
}%
\itKey{p0025}{
May Christ our true God, at the prayers of his most pure and holy Mother; by the power of the precious and life-giving Cross; of the holy, glorious and all-praised Apostles; of Saint N. [the patron of the church]; [of Saint N. whose memory we celebrate, of the holy and righteous forebears of God, Joachim and Anne;] and of all the Saints, have mercy on us and save us, for he is good and loves mankind.
}%
\itKey{p0026}{
On Wednesday
}%
\itKey{p0027}{
May Christ our true God, at the prayers of his most pure and holy Mother; of the holy, glorious and all-praised Apostles; of our Father among the Saints Nicolas of Myra in Lykia, the Wonderworker; of Saint N. [the patron of the church]; [of Saint N., whose memory we celebrate, of the holy and righteous forebears of God, Joachim and Anne;] and of all the Saints, have mercy on us and save us, for he is good and loves mankind.
}%
\itKey{p0028}{
On Friday
}%
\itKey{p0029}{
May Christ our true God, at the prayers of his most pure and holy Mother; of the holy, glorious and all-praised Apostles; of the holy, glorious and triumphant Martyrs; of our venerable and Godbearing Fathers; of Saint N. [the patron of the church]; [of Saint N., whose memory we celebrate; of the holy and righteous forebears of God, Joachim and Anne;] and of all the Saints, have mercy on us and save us, for he is good and loves mankind.
}%
\itKey{p0030}{
NOTE ON THE DISMISSAL
}%
\itKey{p0031}{
According to the strict Typikon only Saints who are ‘feasted’, that is those who have at least one doxastikon at Vespers, should be mentioned in the Dismissal. It is not the moment to read out all the small print from the Synaxarion. There is much variety in the details of the Dismissal in the various books and traditions.
}%
\itKey{p0032}{
GENERAL NOTE ON THE CELEBRATION OF VESPERS
}%
\itKey{p0033}{
The liturgical books assume that the offices are sung by two choirs, that face each other across the church. Choir A, the right-hand choir, stands on the north side of the church and Choir B, the left -hand choir, on the south. Choir A is that of chief singer, or Protopsaltes, and Choir B that of the second singer, or Lampadarios. Choir A normally takes the lead. The letters [A] and [B] at ‘Lord, I have cried’ indicate this. This means that Choir A sings ‘Glory’ and the first Doxastikon and Choir B ‘Both now’ and the first Theotokion. If there is no special Doxastikon, as will frequently be the case on Saturday evening, Choir A sings ‘Glory’, Choir B ‘Both now’ and Choir A the actual Doxastikon/Theotokion.
}%
\itKey{p0034}{
For the singing of the Stichera, it is usual for the chief singer in each choir to take the first Sticheron for their side. Normally the senior person on each side is then asked to sing the second and the third may be offered to a visiting priest or distinguished visitor or singer. Any remaining Stichera are then shared among the other people in each choir.
}%
\itKey{p0035}{
At the Aposticha, Choir B sings the Doxastikon. This means that on Saturday evening Choir B should start the Aposticha, since there are 4 stichera. On Weekdays there are only 3, and so Choir A starts the Aposticha. In this way each Choir sings one of the two Doxastika.
}%
\itKey{p0036}{
In some monasteries the ‘leading’ choir changes at Vespers each Saturday, the cycle beginning at the Matins of Pascha.
}%
\itKey{p0037}{
The Apolytikia are should be sung in the same way by the two Choirs.
}%
\itKey{p0038}{
On Sundays and Feasts, the ‘Resurrection’ Theotokion of the Apolytikion is always sung in the Tone of the immediately preceding Apolytikion. This means that on Saturday evening the Theotokion may not be the one in the Tone of the week.
}%
\itKey{p0039}{
In the monasteries of the Holy Mountain it is customary for the following dialogue to be added at the end of the Dismissal, before the final ‘Through the prayers…’:
}%
\itKey{p0040}{
[A] Accept, Lord, the supplication of us sinners, and have mercy on us.
}%
\itKey{p0041}{
[B] May your mercy, Lord, be upon us, as we have put our in hope.
}%
\itKey{p0042}{
[A] Eternal your memory, blessed and ever remembered Founders.
}%
\itKey{p0043}{
[B] Eternal your memory.
}%
\itKey{p0044}{
THE NINTH HOUR
}%
\itKey{p0045}{
The Ninth Hour ends one liturgical day and Vespers begins the next. They are therefore normally celebrated together.
}%
\itKey{p0046}{
On days when Little Vespers is to be celebrated, the Ninth Hour is read immediately before it.
}%
\itKey{p0047}{
The Ninth Hour should be read in the Narthex, or Liti, if the church has one. If not, it is read in the church, with the Holy Doors and the curtain closed.
}%
\itKey{p0048}{
If there is no Apolytikion or Kontakion given in the Menaion, then those for the ordinary days of the week are used.
}%
\itKey{p0049}{
APPENDIX 2
}%
\itKey{p0050}{
ON BAPTISM
}%
\itKey{p0051}{
The Small Euchologion has an Instruction at the end of the service which runs as follows:
}%
\itKey{p0052}{
Be careful, Priest, to instruct both the nurse and the mother not to bath or wash the face of the newly baptized until the seventh day. On the eighth day they are to wash and bath the child and they are to dispose of the water in a place where no one walks, or in a river, or in the piscina of the Church.
}%
\itKey{p0053}{
It also says that after the service the newly baptized and the Godparent are to go in procession to the former’s house, with everyone carrying lamps and singing As many of you as were baptized into Christ.
}%
\itKey{p0054}{
At the giving of the Cross and the Candle many priests say, for the Cross, If anyone would be my disciple, let them deny themselves, take up their cross and follow me; for the Candle, So let your light shine before your fellows, that they may see your good works, and glorify your Father in heaven.
}%
\itKey{p0055}{
It is the custom in many places for the Godparent to return the newly baptized infant to its mother after the service. The mother should do a prostration before her newly enlightened child and then take it in her arms.
}%
\itKey{p0056}{
If it is not possible for the newly baptized to receive Holy Communion immediately after their Baptism, they should come, or be brought, to the Liturgy at the first opportunity in order to do so. Accompanied by their God-parent they should come with their baptismal candle and be the first of the congregation to receive Communion.
}%