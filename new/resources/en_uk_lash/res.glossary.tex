\itTopic{en}{uk}{lash}{glossary}
\itKey{ltType}{
resource
}%
\itKey{ltTopic}{
glossary
}%
\itKey{ltDomain}{
en_uk_lash
}%
\itKey{p0001}{
GLOSSARY
}%
\itKey{p0002}{
Anavathmi. A series of short hymns, based on the Psalms of Ascents (Greek Anavathmi), nos. 119-133, which are sung at Matins on Sundays, and Feasts with a Gospel. On Sundays they are divided into three Antiphons, except for Tone Eight, which has four. They are sung immediately before the Prokeimenon. On Feasts the first Antiphon of Tone Four is always used, and hence the Prokeimenon on Feasts is always sung in that Tone. Some places use the final hymn from the Second Antiphon in Tone Four, rather than repeat the third one. The latter is more correct, but for convenience we give the alternative as well.
}%
\itKey{p0003}{
Anixantaria. On major feasts, and especially at All-night Vigils, the closing verses of the Opening Psalm are often sung to a solemn and protracted melody. The verses are interpolated with short hymns of praise, such as ‘Glory to you, Holy One. Glory to you, Lord. Glory to you, heavenly King. Glory to you, O God. Alleluia’. The singers take up the Psalm from the verse that begins ‘When you open’, in Greek Anixantos sou, hence the name Anixantaria.
}%
\itKey{p0004}{
Apolytikion. The hymn that precedes the Dismissal (Greek Apolysis). It is the characteristic hymn of the day or the feast, and is often referred to simply as ‘the Troparion of the Day’. It is used at the offices and at the Liturgy.
}%
\itKey{p0005}{
Aposticha. The series of hymns and alternating Psalm verses which are sung towards the end of Vespers and daily Matins. In Greek stichos means ‘verse’.
}%
\itKey{p0006}{
Artoklasia. The ceremony of the Blessing of Loaves, or Breaking (Greek klasis) of Bread (Greek artos), that takes place before the Apolytikion at Vespers when there is a Vigil. In many Greek parishes it is nowadays celebrated on important feasts even when there is no Vigil, at Vespers or Matins or even, incorrectly, after the Liturgy. The ceremony for such occasions will be found in the bilingual edition of the Divine Liturgy published by the Archdiocese of Thyateira.
}%
\itKey{p0007}{
Canon. A series of hymns divided into nine Odes, each linked to one of the nine Biblical Canticles, the last of which consists of the Magnificat and Benedictus. Outside monasteries, apart from the Magnificat, the Biblical Canticles are normally omitted, simple refrains taking the place of the biblical verses. In practice most Canons only contain eight Odes, since Ode 2 is only used on Tuesdays in Lent, though the acrostics often show that there were originally nine. Normally three, on Sundays four, Canons are prescribed, in such a way as to make each Ode, including the Irmos, consist of 14 Troparia.
}%
\itKey{p0008}{
Canonarch. The monk whose task it is to see that the singers sing the correct texts in the correct Tone. He also reads the verses of the Prokeimenon and similar texts. In Slav use these duties have been taken over by the Deacon.
}%
\itKey{p0009}{
Doxology. An ancient hymn of praise, common to both East and West. In the East it has two forms, the Greater and Lesser. The former is used at Matins on Sundays and Feasts, the latter at Matins on ordinary days and at Compline. Like the Latin Te Deum, the hymn itself is followed by a series of verses from the Psalter. The Great Doxology ends with the solemn singing of the Trisagion. The Lesser Doxology ends with the prayer Grant Lord this day, which, at both Vespers and non-festal Matins, precedes the Bowing of Heads. The word ‘doxology’ is also used for other short ascriptions of praise to the Holy Trinity. See Doxastikon and Ekphonesis.
}%
\itKey{p0010}{
Doxastikon. A hymn sung after the short doxology ‘Glory (Greek doxa) to the Father and to the Son and to the Holy Spirit’. It is normally sung to a slower and more elaborate melody than the preceding hymns.
}%
\itKey{p0011}{
Ekphonesis. Used chiefly of the final doxology that ends every prayer. It is usually, but not always, sung out loud (Greek ekphônô), even when the prayer itself is directed to be read ‘in a low voice’ (Greek mystikôs).
}%
\itKey{p0012}{
Eothina. Meaning ‘of the dawn’. The word is used to refer [a] to the eleven Gospels of the Resurrection, one of which is read each Sunday at Matins, and [b] to the eleven idiomels by the Emperor Leo the Wise [866-912], that are sung at the end of Lauds on Sunday. There is one to correspond each of the eleven Gospels, of which they are poetic summaries. In the old ‘Sung’, or ‘Cathedral’, office, as described by St Symeon of Thessaloniki [† 1429], they were sung immediately before the Resurrection Gospel.
}%
\itKey{p0013}{
Epitrachelion. Worn round the neck, which is meaning of the word, it is the Eastern equivalent of the Western stole, the chief difference being that it is always joined down the middle, normally with a series of ornamental studs. It is also broader then the stole. It is the characteristic priestly vestment, worn only by bishops and priests. A priest should not celebrate any service unless he is wearing it.
}%
\itKey{p0014}{
Evlogitaria. A series of short hymns introduced with the refrain ‘Blessed (Greek evlogitos) are you, O Lord, teach me your statutes’, which are sung at the end of Psalm 118, or the Polyeleos, at Matins on Sundays. There is another set which is sung on Saturdays of the Departed, and also at Funerals and Memorial Services.
}%
\itKey{p0015}{
Exapostilarion. Short hymns sung at the end of the Canon at Matins. For Sundays there are eleven, written by the son of Leo the Wise, Constantine Porphyogenitos [912-959], one corresponding to each of the eleven Resurrection Gospels.
}%
\itKey{p0016}{
Gerontika. A convenient word used to indicate those parts of the office which are traditionally read by the Superior, or Elder (Greek Gerôn, or Geronta). If the Superior is absent they are read by the senior monk present. A visiting priest or important visitor is often asked to read them. At festal Matins they include the Six Psalms, Psalm 50 and the concluding Prayer of the First Hour.
}%
\itKey{p0017}{
Ikos. See Kontakion.
}%
\itKey{p0018}{
Irmos. The verse which gives the tune to which the following Ode of a Canon is to be sung. They also frequently make use of, or allude to, the Biblical Canticle which used to be, and on the Holy Mountain is still, sung with the Canon. Normally only the Irmos of the first Canon of a series is sung. On major feasts the Irmi are sung twice and may also be used as Katavasias.
}%
\itKey{p0019}{
Kalymafchion. Also called in monastic language a skoupho, or ‘bonnet’. The familiar ‘stove pot’ hat of Orthodox bishops, priests and deacons. In Greek use those of the secular clergy have a brim on top. Monks, whether priests or not, wear a cowl (Greek koukoulli, from Latin cucullus) over their skoupho. When worn by secular Archimandrites and Bishops it is called an Epanokalymafchion or ‘over kalymafchion’. The Russian cowl is permanently attached to the skoupho, and the whole is called a klobuk, but the Greek cowl can be removed separately, and the rules for when to wear what are quite complicated.
}%
\itKey{p0020}{
Katavasia. The final verse of an Ode of a Canon, so called because the Singers used to come down (Greek katavainein) from their stalls and unite in the middle of the Choir to sing them. On Sundays and major Feasts there is one after every Ode. They are often ‘seasonal’, and anticipate the next great Feast. Thus the Katavasias of Christmas are sung from the 21st of November onwards. The ‘normal’ Katavasias are those of the Canon to the Mother of God in Tone 4, ‘I will open my mouth’. On non-festal days they occur after the 3rd, 6th, 8th and 9th Odes only.
}%
\itKey{p0021}{
Kathisma. One of the twenty sections into which the Psalter is divided for liturgical purposes. The word is also used for the short hymns that are sung after the reading of each Kathisma at Matins. The word is a Greek word meaning a seat. Each Kathisma is divided into three sections (Greek Staseis).
}%
\itKey{p0022}{
Kontakion. Originally a ‘verse sermon’ consisting of a Proemium followed by a series of longer stanzas, or Ikoi, all ending with the same phrase, like a refrain. They were gradually displaced by the Canons, and in the present office only the Proemuim, now called the Kontakion, and the first Ikos survive. In current Greek use they are normally read at Matins, except on great Feasts, and the final phrase is repeated, a survival of the ancient practice. In Russian use the Kontakion is normally sung at Matins on Sundays. The Kontakion is also read at the Hours and sung at the Liturgy, though in Greek use those at the Liturgy are ‘seasonal’, like the Katavasias. The greatest writer of Kontakia is St Romanos [6th century], many of whose Kontakia are still in use.
}%
\itKey{p0023}{
Megalynarion. A short verse containing, either at the beginning or end, the words ‘We magnify’ (Greek megalynomen). On the Feasts of the Lord a series of megalynaria is provided to be sung with the Ninth Ode of the Canon. In Russian use megalynaria are sung with the third part of the Polyeleos. The best known is the one that on most days accompanies the Magnificat and which begins ‘Greater in honour than the Cherubim’.
}%
\itKey{p0024}{
Menaion. From the Greek word for ‘monthly’. The book containing the services for days of the month. There are thus twelve volumes of Menaia. For places without a full set of Menaia there exists in both Greek and Slavonic a volume containing general offices for each category of Saint called the General Menaion. The contents of the Greek and Slavonic General Menaia are not quite the same, the Slavonic containing more offices, as well as texts for a full Vigil for each category of Saint, and also for the Lord and the Mother of God.
}%
\itKey{p0025}{
Metania. A low bow in which the right hand touches the ground. Slavonic poklon. Also used for a prostration.
}%
\itKey{p0026}{
Orarion. The Deacon’s stole. It is worn on the left shoulder and sometimes taken across diagonally under the right arm and again over the left shoulder. It is also worn crossed on the back by readers and sub-deacons.
}%
\itKey{p0027}{
Phelonion. The Eastern equivalent of the Western chasuble. The rubrics direct that the priest is to ‘lower the phelonion’, that is to let it fall over his hands, at the moment of the Dismissal. This indicates that the work of the service is over, rather like rolling down one’s sleeves. Russian phelonia often have a row of buttons across the chest so that the front of the vestment can be raised or lowered.
}%
\itKey{p0028}{
Polyeleos. Psalms 134 and 135 which are sung as the third Kathisma of the Psalter on major Feasts. They are so called because of the many (poly) repetitions of the word ‘mercy’ (eleos) in Psalm 135. The third section consists of selected psalm verses for each major feast. In Russian use these are usually to reduced to one or two, accompanied by a Megalynarion. The word Polyeleos is also the name given to the great chandelier which hangs directly beneath the central dome of the choir. It symbolizes Christ, the ‘most merciful’. When furnished with oil lamps it also uses ‘much oil’ (Greek poly elaion). On the Holy Mountain the Polyeleos is lighted and swung, to spectacular effect, at ‘Lord, I have cried’ and during the Polyeleos at great Vigils. It is also lighted at the Great Entrance on major Feasts, and also at funerals. Around the polyeleos there often hangs a great metal circle, decorated with icons, normally including the twelve Apostles, and surmounted by candles. This is known as a choros.
}%
\itKey{p0029}{
Prokeimenon. A refrain from a Psalm, sung together with one or more verses from the Psalm, that normally precedes the Readings at Vespers, Matins and the Liturgy. Originally the whole Psalm was sung, hence the Verse is normally the first verse of the Psalm. It survives every day at Vespers, even when there are no readings. It is the equivalent of the Western Gradual.
}%
\itKey{p0030}{
Royal Office. The short office that precedes the Six Psalms at Matins. It originated in the monasteries of ‘royal foundation’, as an intercession for the Imperial Family.
}%
\itKey{p0031}{
Sticharion. A tunic-like vestment, resembling the Western dalmatic when worn by servers, readers, sub-deacons and deacons. That of bishops and priests more closely resembles the alb, though it is not necessarily white.
}%
\itKey{p0032}{
Sticheron. A hymn that follows a verse, in Greek stichos, from the Psalms. At Vespers there are always stichera at ‘Lord, I have cried’, but at Lauds at ‘Let everything that has breath, only on Sundays and feasts. At the Aposticha one sticheron is sung before the first psalm verse. The Aposticha at Vespers on Sundays form an alphabetical acrostic, three letters each week, but the first sticheron in each Tone does not form part of the series.
}%
\itKey{p0033}{
Synaxarion. The short notices of the Saints for each day in the Menaion. In most Greek churches the names only are read, together with the doggerel couplet that follows each one, immediately after the Kontakion at Matins.
}%
\itKey{p0034}{
Theotokion. Most series of hymns end with one to the Mother of God, the Theotokos, and so a Theotokion commonly follows the second part of the short doxology, ‘Both now and for ever, and to the ages of ages. Amen’, and is normally sung in the same Tone as the preceding Doxastikon. On Saturday evening, however, the Theotokion at the Entrance is nearly always that of the Tone of the week, regardless of the Tone of the Doxastikon. It is also sung again at Vespers on the following Friday. In Russian usage this Theotokion is called the Dogmatic, whereas in Greek the latter name is used for the corresponding Theotokion at Small Vespers.
}%
\itKey{p0035}{
Typikon. The rules governing the celebration and for combining the different elements of the service. The book containing these rules. It is also used of the ‘rule’ of a monastery. There is a monastic saying, ‘Don’t take your Typikon to another person’s monastery’.
}%
\itKey{p0036}{
Troparion. Any hymn may be called a troparion, but the word more commonly indicates the Apolytikion of the day or one of the stanzas of a Canon. Plural Troparia.
}%
\itKey{p0037}{
Ypakoï. A short hymn that takes the place of the third poetic Kathisma at Matins on Sundays and some greater Feasts. It is normally read and not sung. In Greek use it is also read at the Hours on Sundays in place of the Kontakion. In strict Athonite use it is sung at the Liturgy after the Apolytikia and before the Kontakia.
}%